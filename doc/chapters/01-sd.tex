\chapter{SD Card}
The SD card is connected via SPI and is necessary for normal operation. If no SD-Card is present, no channels can be activated for control. 

File Tree:
\dirtree{%
.1 /.
.2 config/.
.3 control.xml.
.4 name <1-4>.
.4 humidity threshold <1-4>.
.4 state <1-4>.
.4 log\_enable <1-4>.
.3 wifi.xml.
.4 Op-Mode <station/AP>.
.4 SSID.
.4 PassWD.
.4 IP static <optional>.
.2 log/.
.3 plant1.log.
.3 plant2.log.
.3 plant3.log.
.3 plant4.log.
.3 error.log.
}

\vspace{0.5cm}
\textbf{control.xml}

Contains information about the channels and their oepration. Names, humidity thresholds and the states (enabled/disabled) are stored here for the ESP to act upon and to display in the web interface. The log\_enable keys can be used to enable or disable logging for this channel.\\
Default is enabled, errors will always be logged. 

\vspace{0.5cm}
\textbf{wifi.xml}

Contains all necessary information for wifi-usage, like SSID, password and if the ESP should be in station mode (client inside existing network), or in AccessPoint-Mode (creating its own network). Default is AP-Mode.

\vspace{0.5cm}
\textbf{Logfiles}

If logging is enabled, the ADC-values (hourly) and watering-events will be logged. This way long-term data about the moisture and water usage can be obtained for further fine-tuning.